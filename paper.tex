
% Template for ISBI-2013 paper; to be used with:
%          spconf.sty  - ICASSP/ICIP LaTeX style file, and
%          IEEEbib.bst - IEEE bibliography style file.
% --------------------------------------------------------------------------
\documentclass[a4paper,9pt]{extarticle}
\usepackage{spconf,amsmath,amsfonts,amsthm,amssymb,graphicx}
\usepackage{algorithm,algorithmicx,algpseudocode}
\usepackage{times}
\usepackage[font={small}]{caption}
\usepackage{subfigure}
\usepackage{wrapfig}
\usepackage{caption}
\usepackage{transparent}
\usepackage{xcolor,colortbl}
\usepackage{dcolumn}

%% squeezing
\linespread{0.85}
\usepackage[compact]{titlesec}
%\titlespacing{\section}{0pt}{2ex}{1ex}
\titlespacing{\subsection}{0pt}{1ex}{0ex}
\titlespacing{\subsubsection}{0pt}{0.5ex}{0ex}
 
%\def\baselinestretch{0.98}

% Example definitions.
% --------------------
\def\x{{\mathbf x}}
\def\L{{\cal L}}

% Title.
% ------
\title{ A simple and efficient algorithm for computing approximate
  Nash equilibria in sequential games with incomplete
  information}

%
% Single address.
% ---------------
\name{Elvis Dohmatob}
\address{Parietal Team, INRIA, Universit\'e de Paris-Saclay}
%
% For example:
% ------------
%\address{School\\
%	Department\\
%	Address}
%
% Two addresses (uncomment and modify for two-address case).
% ----------------------------------------------------------
%\twoauthors
 %% {A. Author-one, B. Author-two\sthanks{Thanks to XYZ agency for funding.}}
 %%        {School A-B\\
 %%        Department A-B\\
 %%        Address A-B}
 %% {C. Author-three, D. Author-four\sthanks{The fourth author performed the work
 %%        while at ...}}
 %%        {School C-D\\
 %%        Department C-D\\
 %%        Address C-D}
%
% More than two addresses
% -----------------------
% \name{Author Name$^{\star \dagger}$ \qquad Author Name$^{\star}$ \qquad Author Name$^{\dagger}$}
%
% \address{$^{\star}$ Affiliation Number One \\
%     $^{\dagger}$}Affiliation Number Two
%

\newcommand{\fix}{\marginpar{FIX}}
\newcommand{\new}{\marginpar{NEW}}
\DeclareMathOperator{\proj}{proj}
\DeclareMathOperator{\soft}{soft}
\DeclareMathOperator{\prox}{prox}
\DeclareMathOperator{\Prox}{Prox}
\DeclareMathOperator{\im}{im}
\DeclareMathOperator{\rank}{rank}
\DeclareMathOperator{\supp}{supp}

% macros from michael's .tex
\DeclareMathOperator{\dist}{dist} % The distance.
\DeclareMathOperator{\argmin}{argmin}
\DeclareMathOperator{\argmax}{argmax}
\DeclareMathOperator{\Id}{Id}
\DeclareMathOperator{\abs}{abs}
\newcommand{\R}{\mathbb{R}}
\newcommand{\N}{\mathbb{N}}
\newtheorem{thm}{Theorem}[section]
\newtheorem{prop}[thm]{Proposition}
\newtheorem{lemma}[thm]{Lemma}
\newtheorem{corollary}[thm]{Corollary}
\newtheorem{definition}{Definition}
\newtheorem{theorem}{Theorem}
\newtheorem{remark}{Remark}


\begin{document}

\maketitle
\begin{abstract}We present a simple primal-dual algorithm for
  computing approximate Nash equilibria in two-person zero-sum
  sequential games with  incomplete information and perfect recall
  (like Texas Hold'em poker). Our algorithm only performs basic
  iterations (i.e matvec multiplications, clipping, etc., and no calls
  to external first-order oracles, no matrix inversions, etc.)
  and is applicable to a broad class of two-person zero-sum games
  including simultaneous games and sequential games with incomplete
  information and perfect recall. The applicability to the latter kind
  of games is thanks to the sequence-form representation
  \cite{koller1992complexity} which allows us to encode any such game
  as a matrix game with convex polytopial profiles. We prove that
  the number of iterations needed to produce a Nash equilibrium with a
  given precision is inversely proportional to the
  precision. We also present experimental results on simulated and
  real games (Kuhn poker).
\end{abstract}

\textbf{keywords}: algorithmic-game theory; sequential game;
incomplete information; perfect recall; approximate Nash equilibrium;
primal-dual algorithm; convex-optimization

\section{Introduction}
\label{sec:intro}
A game-theoretic approach to playing games strategically optimally
consists of computing Nash-equilibria (in fact, approximations thereof)
offline, and playing one's part (an optimal \textit{behavioral}
strategy) of the equilibrium online. This
technique is the driving-force behind solution concepts like CFR
\cite{zinkevich2008regret,lanctot2009monte,Bowling09012015},
$\text{CFR}^{+}$ \cite{tammelin14} and other variants, which
have recently had profound success in poker. However, solving games
for equilibria remains a mathematical and computational challenge,
especially in sequential games with imperfect information. This paper
proposes a simple and efficient algorithm for solving for such
equilibria approximately, in a sense which will be made clear shortly.

\subsection{Notation and terminology}
Given a set $X$, $2^X$ denotes the \emph{powerset} of $X$, i.e the set
of all subsets of $X$, or equivalently the set of all binary
functions on $X$. Let $m$ and $n$ be positive integers. Given two
vectors $z, w \in \mathbb{R}^n$, their inner product will be denoted
$\langle z, w\rangle := \sum_{j}z_jw_j$. The components of $z$ will be
denoted $z_0$, $z_1$, ..., $z_{n-1}$ (indexing begins from $0$,
not $1$). $\mathbb{R}^{n}_+ := \{z \in \mathbb{R}^{n}\text{ }|\text{ }
z \geq 0\}$ is the nonnegative $n$th \textit{orthant}.
The notation ``$z \ge 0$'' means that all the components of
$z$ are nonnegative.
$\|z\|$ denotes the $2$-\textit{norm} of $z$ defined by $\|z\| :=
\sqrt{\langle z, z\rangle}$. $(z)_+:=\text{max}(0, z) \in
\mathbb{R}^{n}_+$ is the point-wise maximum of $z$ with $0$. For
example, $((-2, \pi))_+ = (max(-2, 0), max(\pi, 0)) = (0, \pi)$. The
operator $(.)_+$ is the well-known (multi-dimensional) \textit{ramp}
function. The $n$-simplex denoted $\Delta_n$, is defined by $\Delta_n
:= \{z \in \mathbb{R}^n_+|\sum_j z_j = 1\}$.
Given a matrix $A \in \mathbb{R}^{m \times n}$, its \textit{spectral
  norm}, denoted $\|A\|$, is
 defined to be the largest \textit{singular value} of $A$, i.e the
 largest \textit{eigenvalue} of $A^TA$ (or equivalently, of $AA^T$).

Let us introduce some basic but powerful modern convex-analytical
notions which will be essential in the sequel. Given a subset $C$ of
$\mathbb{R}^n$, $i_C$ denotes its \textit{indicator function} defined
by $i_C(x) = 0 \text{ if } x \in C\text{ and }+\infty\text{
    otherwise}$.
At times, we will write $i_{x \in C}$ for $i_C(x)$ (to ease notation,
etc.). For example, we will write $i_{z \ge 0}$ for
$i_{\mathbb{R}^n_+}(z)$, etc. Let $f : \mathbb{R}^n \rightarrow
(-\infty, +\infty]$ be a
  convex function. The \textit{orthogonal projector} onto $C$, is the
  ``closest-point function'' $\proj_C: \mathbb{R}^n \rightarrow C, x
  \mapsto \underset{z \in
      C}{\text{argmin }}\frac{1}{2}\|z-x\|^2$.
The \textit{effective domain} of $f$, denoted
  $dom(f)$, is defined as
$dom(f) := \{x \in \mathbb{R}^n | f(x) < +\infty\}$.
 If $dom(f) \ne \emptyset$ then we say $f$ is \textit{proper}.
%% \textit{proper convex lower semi-continous function}
%% (\textit{p.c.l.s.c} for short).
%% The \textit{Fenchel-Legendre transform} of $f$ is the function $f^*:
%% \mathbb{R}^n \rightarrow (-\infty, +\infty]$ defined by $f^*(x) \equiv
%%   \underset{z \in \mathbb{R}^n}{\text{max}}\text{}z^Tx - f(z)$.
The \textit{subdifferential} of $f$ at a point $x \in \mathbb{R}^n$ is
defined by
$\partial f(x) := \{s\in \mathbb{R}^n | f(z)  \ge f(x) + \langle s, z -
x\rangle, \forall z \in \mathbb{R}^n\}$.
%% Each element of $\partial f(x)$ is called a \textit{subgradient} of $f$
%% at $x$. Of course $\partial f(x)$ reduces to the singleton $\{\nabla f(x)\}$
%% in case $f$ is differentiable at $x$.
If $f$ is convex, its
\textit{proximal operator} is the function $\prox_f: \mathbb{R}^n
\rightarrow \mathbb{R}^n$ defined by $\prox_f(x): = \underset{z \in
  \mathbb{R}^n}{\text{min }}\frac{1}{2}\|z
  - x\|^2 + f(z)$.
%% For example, if $C$ is a closed convex subset of $\mathbb{R}^n$, then
%% $\prox_C = \proj_C$, the orthogonal projector onto $C$. Thus proximal
%% operators generalize orthogonal projectors. For example
%% $\prox_{i_{\mathbb{R}^n_+}}(z) \equiv \proj_{\mathbb{R}^n_+}(z) \equiv
%% (z)_+$. One also has the useful characterization
%% \begin{eqnarray}
%%   p = \prox_f(x)\text{ iff } x - p \in \partial f(p).
%% \end{eqnarray}
%% Thus given $\gamma > 0$, the operator $\prox_{\gamma f}$ can (and
%% should) be thought of as performing an \textit{implicit} gradient step
%% of size $\gamma$.

The interested reader should refer to \cite{combettes2011proximal} and
the references therein, for a more elaborate exposition on
proximal calculus and its use in modern convex-optimization.

\subsection{Statement of the problem}

The sequence-form representation for two-person zero-sum games with
incomplete information was introduced in
\cite{koller1992complexity}, and the theory was further developed in
\cite{koller1994fast,von1996efficient,vonequilibrium} where it was
established that for such games, there exist sparse matrices
$A \in \mathbb{R}^{n_1 \times n_2}$, $E_1 \in \mathbb{R}^{l_1 \times
  n_1}$, $E_2 \in \mathbb{R}^{l_2 \times n_2}$, and vectors $e_1 \in
\mathbb{R}^{l_1}, e_2 \in \mathbb{R}^{l_2}$ such that $n_1$, $n_2$,
$l_1$, and $l_2$ are all linear in the size of the game tree (number
of states in the game) and such that Nash equilibria correspond to
pairs $(x, y)$ of \textit{realization plans} which solve the primal
LCP (Linear Convex Program)
\begin{equation}
  \begin{aligned}
    \underset{(y,p) \in \mathbb{R}^{n_2} \times
     \mathbb{R}^{l_1}}{\text{minimize }}\langle e_1,
    p\rangle \hspace{.5em}\text{
       subject to: } &y \ge 0, E_2y = e_2,\\
    &-Ay + E_1^Tp \geq 0,
  \end{aligned}
  \label{eq:primal_pb}
\end{equation}

and the dual LCP
\begin{equation}
  \begin{aligned}
    \underset{(x,q) \in \mathbb{R}^{n_1} \times
      \mathbb{R}^{l_2}}{\text{maximize }}-\langle e_2,
    q\rangle\hspace{.5em}\text{subject
      to: } &x \ge
    0, E_1x = e_1,\\
    &A^Tx + E_2^Tq \geq 0.
  \end{aligned}
  \label{eq:dual_pb}
\end{equation}
The vectors $p = (p_0, p_1, ..., p_{l_2 - 1}) \in \mathbb{R}^{l_2}$
and $q = (q_0, q_1, ..., q_{l_1 - 1}) \in \mathbb{R}^{l_1}$ are dual
variables. 
$A$ is the \textit{payoff matrix} and each $E_k$ is a matrix whose
entries are $-1$, $0$ or $1$, with exactly 1 entry per row which
equals $-1$ except for the first whose whose first entry is $1$ and all
the others are $0$. Each of the vectors $e_k$ is of the form $(1, 0, ..., 0)$.

Note that the LCPs above have the equivalent saddle point formulation
\begin{equation}
  \underset{y \in Q_2}{\text{minimize}}\text{ }\underset{x \in
    Q_1}{\text{maximize}}\text{ }\langle x, Ay\rangle,
  \label{eq:gilpin}
\end{equation}
where the \textit{convex polytope} (by which we mean a bounded convex
polyhedron)
\begin{equation}
  Q_k := \{z \in \mathbb{R}^{n_k}_+ |\text{ }E_kz = e_k\} \subseteq
  [0, 1]^{n_k}
\label{eq:polytope}
\end{equation}
is identified with the strategy profile of player $k$ in the
sequence-form representation. At a feasible point $(y, p, x, q)$ for
the LCPs, the \textit{primal-dual gap} $\tilde{G}(y, p, x, q)$ is
given by\footnote{The first inequality being due to \textit{weak
    duality}.}
\begin{eqnarray}
  \begin{split}
  0 &\le \tilde{G}(y, p, x, q) := \langle e_1, p\rangle - (-\langle
  e_2, q\rangle) = \langle e_1, p\rangle + \langle
  e_2, q\rangle\\
  &= G(x, y) := \mathrm{max}\{\langle u, Ay\rangle - \langle x, Av\rangle |
(u,v) \in Q_1 \times Q_2\}.
\end{split}
  \label{eq:dgap}
\end{eqnarray}

In \eqref{eq:dgap}, the quantity $G(x, y)$ is nothing but the primal-dual
gap for the equivalent saddle point problem \eqref{eq:gilpin}.
It was shown (see Theorem 3.14 of \cite{vonequilibrium}) that a pair
$(x, y) \in Q_1 \times Q_2$ of realization plans is a solution to the
LCPs \eqref{eq:primal_pb} and \eqref{eq:dual_pb} (i.e is a Nash
equilibrium for the game)  if and only if there exist vectors $p$ and
$q$ such that
\begin{equation}
\begin{split}
\hspace{.25em} -Ay + E_1^Tp \ge 0, &\hspace{.5em}A^Tx + E_2^Tq \ge
0, \hspace{.25em} \langle x, -Ay + E_1^Tp\rangle = 0,\\
&\langle y, A^Tx  + E_2^Tq\rangle = 0.
\end{split}
\label{eq:feasibility}
\end{equation}

Moreover, at equilibria, \textit{strong duality} holds and the value
of the game equals $p_0 = -q_0$, i.e the primal-dual gap
$\tilde{G}(y, p, x, q)$ defined in \eqref{eq:dgap} vanishes at
equilibria.

Solving the LCPs \eqref{eq:primal_pb} and \eqref{eq:dual_pb} exactly
is impossible in practice (indeed, this system of problems is NP-hard
\cite{koller1992complexity}) and such a precision doesn't have any
fundamental practical advantage. Instead, it is customary compute
``approximate'' Nash equilibria. A popular notion of approximate
equilibria is the following:

\begin{definition}[\textbf{Nash $\epsilon$-equilibria}]
Given $\epsilon > 0$, a Nash $\epsilon$-equilibrium is
a pair $(x^*, y^*)$ of realization plans such that there exists dual
vectors $p^*$ and $q^*$ for problems \eqref{eq:primal_pb} and
\eqref{eq:dual_pb} such that the primal-dual gap at $(y^*, p^*, x^*, q^*)$
doesn't exceed $\epsilon$. That is,

\begin{equation}
  0 \le \tilde{G}(y^*, p^*, x^*, q^*) \le \epsilon.
\label{eq:approx_pb}
\end{equation}
\label{thm:approx_nash}
\end{definition}

\begin{remark}
It should be noted
that any matrix $A \in \mathbb{R}^{n_1 \times n_2}$ specifies a matrix
  game with payoff matrix $A$, for which each player's strategy
profile is a simplex; this simplex can be written in the form
\eqref{eq:polytope} by taking $E_k := (1, 1, ..., 1) \in
\mathbb{R}^{1 \times n_k}$ and $e_k = 1 \in \mathbb{R}^1$. Thus every
matrix game on simplexes can be seen as a sequential game.  Thus the
results presented in this manuscript can be trivially applied such
games in particular. Here, the polytopial $Q_k$
defined in \eqref{eq:polytope} reduce to simplexes $\Delta_{n_k}$,
and the primal-dual gap function $G(x,y)$ writes
\begin{eqnarray}
\begin{split}
G(x, y) &=
\mathrm{max}\{\langle u, Ay\rangle - \langle x, Av\rangle | (u,v) \in
\Delta_{n_1} \times \Delta_{n_2}\}\\
&= \underset{0 \le i <
  n_1}{\text{max }}(Ay)_i - \underset{0 \le j < n_2}{\text{min
}}(A^Tx)_j.
\end{split}
\label{eq:mg_pd}
\end{eqnarray}
\end{remark}

\subsection{Quick sketch of our contribution}
One cannot directly attack the LCPs \eqref{eq:primal_pb} and
\eqref{eq:dual_pb} via a traditional primal-dual algorithm
(for example \cite{chambolle2010,chambolle2014ergodic}) because
computing the orthogonal projections $\proj_{Q_k}$ is very difficult
In fact, such sub-problems would have to be solved
iteratively\footnote{An ``exception to the rule'' is the case where
  the $Q_k's$ are simplexes, so that the projections can be computed
  exactly using \cite{duchi2008efficient}.}. Also,
the primal-dual gap might explode even at points arbitrarily close to
the set of feasible points, leaving the algorithm with no indication
whatsoever, on whether progress is being made or not. So need to way to
\begin{itemize}
\item[(1)] avoid having to compute the projections $\proj_{Q_k}$,
\item[(2)] have control over how far we are from the set of
  equilibria and avoid infinite (and thus non-informative) primal-dual
  gaps.
\end{itemize}

Developing on an alternative notion of approximate equilibria (see
Definition \ref{thm:cool_notion})
homologous to that presented in Definition \ref{thm:approx_nash}, we
device a primal-dual algorithm that (Algorithm \ref{Tab:algo}) for
computing approximate Nash equilibria in sequential two-person
zero-sum games with incomplete information and perfect recall, and
which satisfies the above crucial requirements (1)--(2). We also
prove (Theorem \ref{thm:pd}) that --in an ergodic / Ces\`ario sense-- the number of
iterations required by the algorithm to produce an approximation
equilibrium to a precision $\epsilon$ is $\mathcal{O}(1/\epsilon)$,
with explicit values for the constants involved this worst-case cost.
These contributions will be elaborated in section \ref{sec:gsp}.
%%  The source of approximation in our scheme
%% is that we explicitly control how close we are to the feasible set
%% $Q_1 \times Q_2$.


\section{Related work}
\label{sec:related_work}
We now present a selection of algorithms that is representative of the
efforts that have been made in the literature to compute Nash
$\epsilon$-equilibria for two-person zero-sum games with incomplete
information like Texas Hold'em poker, etc.
First and foremost, let us note that for the class of games considered
here (sequential games with incomplete information), the LCPs
\eqref{eq:primal_pb} and \eqref{eq:dual_pb} are exceedingly larger
than what state-of-the-art LCP and interior-point solvers can
handle. See for example \cite{hoda2010smoothing,gilpinfirst}.

In \cite{hoda2010smoothing}, a nested iterative procedure using the
Excessive Gap Technique (EGT) \cite{nesterov2005excessive} was used
to solve the equilibrium problem \eqref{eq:gilpin}.
The authors reported a $\mathcal{O}(1/\epsilon)$ convergence rate
(which derives from the general EGT theory) for the outer-most
iteration loop.
\cite{gilpinfirst} proposed a modified version of the techniques in
\cite{hoda2010smoothing} and  proved a $\mathcal{O}\left(\left(\|A\| /
\delta\right) ln\left(1 / \epsilon\right)\right)$ convergence rate in
terms of the number of calls made to a first-order oracle. Here
$\delta = \delta(A, E_1, E_2, e_1, e_2) > 0$ is a certain
\textit{condition number} for the game. The crux of their technique was to
observe that \eqref{eq:gilpin} can further be written a the minimization of
the primal-dual gap function $G(x, y)$ (defined in \eqref{eq:dgap})
for the game\footnote{The minimizers of $G$ are precisely the
  equilibria of the game.}, viz
\begin{eqnarray}
\mathrm{minimize}\{G(x,y)|(x,y) \in Q_1 \times Q_2\},
\end{eqnarray}
and then show there exists a scalar
$\delta > 0$ such that for any pair of realization plans $(x, y) \in Q_1 \times Q_2$,
\begin{eqnarray}
\text{``distance between }(x, y)\text{ and the set of
equilibria'' } \le G(x, y)/\delta.
\end{eqnarray}
Their
algorithm is then derived by iteratively applying Nesterov smoothing \cite{nesterov2005a}
with a geometrically decreasing sequence of tolerance levels
$\epsilon_{n+1} = \epsilon_n / \gamma$ (with $\gamma > 1$)  $G$. It
should be noted however that
\begin{itemize}
\item[--] The constant $\delta > 0$ can be arbitrarily small, and so
  the factor $\|A\| / \delta$ in the $\mathcal{O}\left(\left(\|A\| /
\delta\right) ln\left(1 / \epsilon\right)\right)$ convergence rate can
be arbitrarily large for ill-conditioned games.
\item[--] The reported linear convergence rate is not in terms of
  basic operations (addition, multiplication, matvec, clipping, etc.),
  but in terms of the number of calls to a first-order oracle. Most
  notably, the complicated projections $\proj_{Q_k}$ are applied at
  each iteration.%%  Of
  %% course, the existence of a linear algorithm for solving
  %% \eqref{eq:gilpin} is improbable, as the usual strong-convexity
  %% conditions are absent.

\end{itemize}

%% It should be noted that the EGT and its precursors have had
%% considerable success in the signal processing communities, as can
%% be seen in \cite{NestaCandès, eduard, etc.} and the references
%% therein.


The primal-dual algorithm first developed in \cite{chambolle2010}, was
proposed in \cite{chambolle2014ergodic} as a way of solving matrix
games on simplexes. It should be stressed that such matrix games on
simplexes are considerably simpler than the games considered
here. Indeed, the authors in \cite{chambolle2014ergodic} used the fact
that computing the orthogonal projection of a point onto a simplex can
be done in linear time as in \cite{duchi2008efficient}. In contrast,
no such efficient algorithm is known nor is likely to exist, for the
polytopes $Q_k$ defined in \eqref{eq:polytope}.%% , the strategy
%% profiles for players in %% the games considered here.
That notwithstanding, such projections can still be done iteratively
using for example, the algorithm in proposition 4.2 of
\cite{combettes2010dualization} or the algorithms developed in
\cite{tran2015splitting}. Unfortunately, as with any nested iterative
scheme, one would have to solve this sub-problem with finer and finer
precision.


CFR (CounterFactual Regret minimization) \cite{zinkevich2008regret},
Monte Carlo CFR \cite{lanctot2009monte}, and CFR+
\cite{Bowling09012015} have also become state-of-the-art, and are
particularly useful in many-player games, since convex-analytical
methods cannot help much in such games.


\section{Methods}
\label{sec:gsp}
\subsection{The Generalized Saddle-point Problem (GSP) connection}
In the next theorem, we show that the LCPs \eqref{eq:primal_pb}
and \eqref{eq:dual_pb} can be
conveniently written as a \textit{Generalized Saddle-point Problem
  (GSP)} in the sense of \cite{he2013accelerating}. The crux of idea
is to remove the linear constraints in the definitions of the strategy
polytopes $Q_k$, by modifying the payoff matrix to yield an equivalent
saddle-point problem.
\begin{theorem}
Define two proper closed convex functions
  \begin{eqnarray}
    \left.
    \begin{aligned}
      g_1: \mathbb{R}^{n_2} &\times \mathbb{R}^{l_1} \rightarrow
      (-\infty, +\infty], \hspace{1em} g_1(y, p) :=
        i_{y \ge 0} + \langle e_1,p\rangle\\
        g_2: \mathbb{R}^{n_1} &\times \mathbb{R}^{l_2} \rightarrow
        (-\infty, +\infty],\hspace{1em} g_2(x, q) :=
          i_{x \ge 0} + \langle e_2, q\rangle
    \end{aligned}
    \right\}
    \label{eq:things}
  \end{eqnarray}

Also define two bilinear forms $\Psi_1$, $\Psi_2: \mathbb{R}^{n_2}
\times \mathbb{R}^{l_1} \times \mathbb{R}^{n_1} \times
\mathbb{R}^{l_2} \rightarrow \mathbb{R}$ with $\Psi_2 = -\Psi_1$ by letting
\begin{equation}
  \begin{split}  
    K :=
    \left[
      \begin{array}{cc}
        A & -E_1^T \\
        E_2 & 0
      \end{array}
      \right],\hspace{.5em}
    \Psi_1(y, p, x, q)
    := \left\langle \begin{bmatrix}x\\q\end{bmatrix},
      K\begin{bmatrix}y\\p\end{bmatrix}\right\rangle%% \\
      %% &= \langle x,
      %% Ay\rangle -\langle x, E_1^Tp\rangle + \langle q, E_2y\rangle,
\end{split}
\end{equation}

and define the functions $\hat{\Psi}_1$, $\hat{\Psi}_2:
\mathbb{R}^{n_2} \times \mathbb{R}^{l_1} \times \mathbb{R}^{n_1}
\times \mathbb{R}^{l_2} \rightarrow (-\infty, +\infty]$ by
\begin{eqnarray}
  \begin{aligned}
    \hat{\Psi}_1(y, p, x, q) :=\begin{cases}
    \Psi_1(y, p, x, q)+ g_1(y, p), &\mbox{ if }y \ge 0,\\
    +\infty, &\mbox{ otherwise}\end{cases}\\
    %% &\hat{\Psi}_2(y, p, x, q) &:= \Psi_2(y, p, x, q)+ g_2(x, q) \text{ if
    %% }x \ge 0, \hat{\Psi}_2(y,p, x, q) := \infty \text{ otherwise}.
    \hat{\Psi}_2(y, p, x, q) :=\begin{cases}
    \Psi_2(y, p, x, q)+ g_2(y, p), &\mbox{ if }x \ge 0,\\
    +\infty, &\mbox{ otherwise}\end{cases}
  \end{aligned}
\end{eqnarray}

Finally, define the sets $S_1 := \mathbb{R}^{n_2}_+ \times \mathbb{R}^{l_1}$ and $S_2 :=
\mathbb{R}^{n_1}_+ \times \mathbb{R}^{l_2}$ and consider the
GSP($\Psi_1$, $\Psi_2$, $g_1$, $g_2$): Find a quadruplet $(y^*,p^*, x^*, q^*) \in
S_1 \times S_2$ such that $\forall (y,p, x, q) \in S_1
    \times S_2$, it holds that
\begin{eqnarray}
  \begin{split}
    &\hat{\Psi}_1(y^*, p^*, x^*, q^*) \le \hat{\Psi}_1(y, p, x^*,
    q^*),\\
    &\hat{\Psi}_2(y^*, p^*, x^*, q^*)
    \le \hat{\Psi}_2(y^*, p^*, x, q).
  \label{eq:unconstrained_pb}
\end{split}
\end{eqnarray}
\label{thm:pd}

Then GSP($\Psi_1$,
  $\Psi_2$, $g_1$, $g_2$) is equivalent to the Nash equilibrium LCPs
  \eqref{eq:primal_pb} and \eqref{eq:dual_pb}, i.e
a quadruplet $(y^*,p^*, x^*, q^*) \in \mathbb{R}^{n_2}
  \times \mathbb{R}^{l_1} \times \mathbb{R}^{n_1} \times
  \mathbb{R}^{l_2}$ solves the Nash equilibrium LCPs
  \eqref{eq:primal_pb} and \eqref{eq:dual_pb} iff it solves
  GSP($\Psi_1$, $\Psi_2$, $g_1$, $g_2$). 
  \label{thm:pd}
\end{theorem}

\begin{proof}
It suffices to show that at any point $(y, p, x,
q) \in S_1 \times S_2$, the primal-dual gap between the primal
LCP \eqref{eq:primal_pb} and the dual LCP \eqref{eq:dual_pb} equals
the primal-dual gap of GSP($\Psi_1$, $\Psi_2$, $g_1$, $g_2$).
Indeed, the unconstrained objective in \eqref{eq:primal_pb}, say
$a(x,y)$, can be computed as
\begin{eqnarray*}
  \begin{aligned}
    &a(y,p) = \langle e_1,p\rangle + i_{y\ge 0} + i_{-Ay + E_1^Tp \ge 0} +
    i_{E_2y = e_2}\\
    &= g_1(y,p) + \underset{x' \geq
      0}{\text{max}}\text{ }\langle x',Ay - E_1^Tp\rangle +
    \underset{q'}{\text{max}}\text{ }\langle q',E_2y - e_2\rangle\\
    &= g_1(y,p) + \underset{x',
      q'}{\text{max}}\text{ }\langle x',Ay\rangle - \langle x',
    E_1^Tp\rangle + \langle q',E_2y\rangle\\
    &\hspace{10em}-
    (i_{x' \ge 0} + \langle e_2,q\rangle)\\
    &= g_1(y,p)
      - \underset{x',q'}{\text{min}}\text{ }\Psi_2(y, p, x', q') + g_2(x',
      q')\\
      &= g_1(y,p)
      - \underbrace{\underset{x',q'}{\text{min}}\text{
        }\hat{\Psi}_2(y, p, x', q')}_{\phi_2(y,p)}
      = g_1(y, p) - \phi_2(y, p).
  \end{aligned}
  \label{eq:a}
\end{eqnarray*}

Similarly, the unconstrained objective, say
$b(x, q)$, in the dual LCP \eqref{eq:dual_pb} writes
\begin{eqnarray*}
  \begin{aligned}
&b(x, q) = 
-\langle q, e_2\rangle -i_{x \ge 0} - i_{A^Tx+E_2^Tq \ge 0} -
 i_{E_1x = e_1}\\
 &= -g_2(x, q) + \underset{y' \geq
   0}{\text{min}}\text{ }\langle y', A^Tx + E_2^Tq\rangle +
 \underset{p'}{\text{min}}\text{ }\langle p', e_1-E_1x\rangle\\
    &= -g_2(x, q)
 +\underset{y',p'}{\text{min}}\text{ }\Psi_1(y', p', x, q) +
 g_1(y', p')\\
& = -g_2(x, q) +
 \underbrace{\underset{y',p'}{\text{min}}\text{ }\hat{\Psi}_1(y', p',
   x, q)}_{\phi_1(x, q)} = -g_2(x, q) + \phi_1(x, q). 
   \end{aligned}
\end{eqnarray*}
Thus, noting that $-\infty \le \phi_1(x, q), \phi_2(y, p), < +\infty$
(so that all the operations below are valid),
one computes the primal-dual gap between the primal LCP
\eqref{eq:primal_pb} and dual the LCP \eqref{eq:dual_pb} at $(y, p, x, q)$ as
\begin{eqnarray*}
  \begin{split}
    &a(y, p) - b(x, q) = g_1(y, p) - \phi_2(y, p) + g_2(x, q) - \phi_1(x,
  q) \\
  &= \Psi_1(y, p, x, q) +  g_1(y, p) - \phi_2(y, p) + \Psi_2(y, p, x,
  q) + g_2(x, q) \\
  &\hspace{2em}- \phi_1(x, q)\\
  &= \hat{\Psi}_1(y, p, x, q) + \hat{\Psi}_2(y, p, x, q) - \phi_1(x,
  q) - \phi_2(y, p)\\
  &= \text{primal-dual gap of GSP}(\Psi_1, \Psi_2,
  g_1, g_2) \text{ at }(y, p, x, q),
  \end{split}
\end{eqnarray*}
where the second equality follows from the zero-sum condition $\Psi_1
+ \Psi_2 := 0$.
\end{proof}


By Theorem \ref{thm:pd}, solving for a Nash equilibrium for
the game is equivalent to solving the GSP
\eqref{eq:unconstrained_pb}, which as it turns out, is simpler
conceptually (for example, we no longer need to compute the
complicated orthogonal projections $\proj_{Q_k}$). The rest of the
paper will be devoted to developing an algorithm for solving the latter.


% \subsection{The proposed algorithm and its $\mathcal{O}(1/\epsilon)$
% convergence}

\subsection{A more comfortable notion of approximate Nash equilibrium}
\begin{definition}
  Given a scalar $\epsilon > 0$ and a function $f:\mathbb{R}^n
  \rightarrow [-\infty,+\infty]$, the $\epsilon$-enlarged
  subdifferential (or
  $\epsilon$-subdifferential, for short) of $f$ is the set-valued
  function defined by
  \begin{eqnarray}
\partial_\epsilon f(x):= \{s \in \mathbb{R}^n | f(z)
\ge f(x) + \langle s, z - x\rangle - \epsilon,\forall z \in
\mathbb{R}^n\}.
\end{eqnarray}
\end{definition}
The idea behind $\epsilon$-subdifferentials is the following. Say we wish
to minimize the function $f$. Replace the usual necessary
condition ``$0 \in \partial f(x)$'' for the optimality of $x$ with the
weaker condition ``$\partial_\epsilon f(x)$ contains a sufficiently
small vector $v$''. In fact, it is easy to see that, for each
point $x \in \mathbb{R}^n$, we have $\underset{\epsilon \rightarrow
  0^+}{\text{lim }}\partial_\epsilon f(x) = \partial f(x)$.

This approximation concept for subdifferentials yields the following
concept of approximate Nash equlibria (adapted from
\cite{he2013accelerating}).

\begin{definition}[\textbf{Nash $(\epsilon_1,\epsilon_2)$-equilibria}]
Given tolerance levels $\epsilon_1, \epsilon_2 > 0$, a Nash
$(\epsilon_1,\epsilon_2)$-equilibrium for the GSP \eqref{eq:unconstrained_pb}
is any quadruplet $(x^*, y^*, x^*, q^*)$ for which
there exists a perturbation vector $v^*$ such that
$\|v^*\| \le \epsilon_1$ and $v^* \in
\partial_{\epsilon_2}[\hat{\Psi}_1(., ., x^*, q^*) +
  \hat{\Psi}_2(y^*, p^*, ., .)](y^*,p^*,x^*,q^*)$.

Such a vector $v^*$ is called a Nash $(\epsilon_1,
\epsilon_2)$-residual at the point $(x^*,
y^*, x^*, q^*)$.
\label{thm:cool_notion}
\end{definition}


The above definition is a generalization of the notion of Nash
equilibria since:
\begin{itemize}
\item Exact Nash equilibria correspond to Nash $(0,0)$-equilibria.
\item Nash $\epsilon$-equilibria (in the sense of Definition
  \ref{thm:approx_nash}) correspond to Nash $(0,\epsilon)$-equilibria.
\end{itemize}


\subsection{The proposed algorithm}
\label{sec:algo}
We now derive the algorithm (Algorithm \ref{Tab:algo}) which is the
  main object of this
manuscript, and establish the theoretical properties.

\begin{algorithm}
\caption{Primal-dual algorithm for computing approximate Nash
  Equilbria in two-person zero-sum games with incomplete information
  and perfect recall}
\label{Tab:algo}
\begin{algorithmic}[1]
\Require $\epsilon > 0$; $(y^{(0)},p^{(0)},x^{(0)},q^{(0)}) \in \mathbb{R}^{n_2}
  \times \mathbb{R}^{l_1} \times \mathbb{R}^{n_1} \times
  \mathbb{R}^{l_2}$.
\Ensure A Nash $(\epsilon,0)$-equilibrium
$({y^*},{p^*},{x^*},{q^*}) \in S_1 \times S_2$ for
the GSP \eqref{eq:unconstrained_pb}.
\State  $\lambda \leftarrow 1/\|K\|$, ${v}^{(0)} \leftarrow 0$, $k
\leftarrow 0$
\While{ $ k = 0$ or  $\frac{1}{k\lambda}\|v^{(k)}\| \ge \epsilon$}
\State $y^{(k + 1)} \leftarrow (y^{(k)} - \lambda (A^Tx^{(k)} +
E_2^Tq^{(k)}))_+$, \hspace{.5em}$p^{(k+1)} \leftarrow p^{(k)} -
\lambda(e_1-E_1x^{(k)})$
\State $x^{(k + 1)} \leftarrow (x^{(k)} + \lambda (Ay^{(k+1)} -
E_1^Tp^{(k+1)}))_+$, \hspace{.5em}$\Delta x^{(k+1)} \leftarrow
x^{(k+1)}-x^{(k)}$
\State $\Delta q^{(k+1)} \leftarrow \lambda (E_2y -
e_2)$, \hspace{.5em}$q^{(k+1)} \leftarrow q^{(k)} + \Delta q^{(k+1)}$
\State $y^{(k+1)} \leftarrow y^{(k+1)} - \lambda (A^T\Delta x^{(k+1)}
+ E_2^T\Delta q^{(k+1)})$, \hspace{.5em}$\Delta y^{(k+1)} \leftarrow
y^{(k+1)}-y^{(k)}$
\State $p^{(k+1)} \leftarrow p^{(k+1)} + \lambda E_1\Delta x^{(k+1)}$,
\hspace{.5em} $\Delta p^{(k+1)} \leftarrow p^{(k+1)}-p^{(k)}$
\State ${v}^{(k+1)} \leftarrow {v}^{(k)} + (\Delta
y^{(k+1)},\Delta p^{(k+1)},\Delta x^{(k+1)},\Delta q^{(k+1)})$ 
\State $k \leftarrow k + 1$
\EndWhile
\end{algorithmic}
\end{algorithm}

\begin{theorem}[Ergodic / Ces\`ario $\mathcal{O}(1/\epsilon)$ convergence]
Let $d_0$ be the orthogonal distance between the starting point
$(y^{(0)},p^{(0)},x^{(0)},q^{(0)})$ of Algorithm \ref{Tab:algo} and the
set of equilibria for the GSP \eqref{eq:unconstrained_pb}.
Then given any $\epsilon > 0$, there exists an index
$k_0 \le \frac{2d_0\|K\|}{\epsilon}$ such that after $k_0$ iterations
the algorithm produces a quadruplet
$(y^{k_0},p^{k_0},x^{k_0},q^{k_0})$ and a vector $v^{k_0}$ such that
$\|v_a^{k_0}\| \le \epsilon$ and $v_a^{k_0} \in
\partial[\hat{\Psi}_1(., ., x^{k_0}, q^{k_0}) +
  \hat{\Psi}_2(y^{k_0}, p^{k_0}, ., .)](y^{k_0},p^{k_0},x^{k_0},q^{k_0})$,
where $v_a^{(k_0)} := \frac{1}{k\lambda}v^{(k_0)}$.
Thus Algorithm \ref{Tab:algo} outputs an $(\epsilon,0)$-Nash
equilibrium for the GSP \eqref{eq:unconstrained_pb}
in at most $\frac{2d_0\|K\|}{\epsilon}$ iterations.
\end{theorem}

\begin{proof}
It is clear to see that the quadruplet $(\Psi_1, \Psi_2, g_1, g_2)$
satisfies assumptions B.1, B.2, B.3, B.5, and B.6 of
\cite{he2013accelerating} with $L_{xx} = L_{yy} = 0$ and $L_{xy} =
L_{yx} = \|K\|$. Now, one easily computes the proximal operator of
$g_j$ in closed-form as $\prox_{\lambda g_j}(a, b) \equiv ((a)_+,
b - \lambda e_j)$. With all these ingredients in place, Algorithm
\ref{Tab:algo} is then obtained from \cite[Algorithm
  T-BD]{he2013accelerating}, applied on the GSP
\eqref{eq:unconstrained_pb} with the choice of parameters: $\sigma = 1
\in (0, 1]$, $\sigma_x = \sigma_y = 0 \in [0, \sigma)$,
    $\lambda_{xy} := \frac{1}{\sigma L_{xy}}\sqrt{(\sigma^2 -
        \sigma_x^2)(\sigma^2 - \sigma_y^2)} = \sigma / \|K\| =
      1/\|K\|$, and $\lambda = \lambda_{xy} \in (0,
      \lambda_{xy}]$. The convergence result follows immediately from
  \cite[Theorem 4.2]{he2013accelerating}.
\end{proof}

\subsection{Practical considerations}
\paragraph*{\textbf{\textit(a) Efficient computation of $Ay$ and $A^Tx$.}}
In Algorithms \ref{Tab:algo}, most of the time is spent
pre-multiplying vectors by $A$ and $A^T$ (precisely 3 such operations
are done per iteration). For \textit{flop-type} poker
games like \textit{Texas Hold'em} and  \textit{Rhode Island Hold'em},
$A$ (and thus $A^T$ too)  is very big (up $10^{14}$ rows and columns!)
but has a rich block-diagonal structure (each block is itself the
Kronecker product of smaller matrices) which can be carefully
exploited, as was done in \cite{hoda2010smoothing}.%%  For the purpose of
%% completness, we explain the details of the tricks used in
%% \cite{hoda2010smoothing} for speeding up the computation of these
%% matvec products.

%% \begin{definition}(Kronecker product)
%% Let $F \in \mathbb{R}^{m \times n}$ and $B \in \mathbb{R}^{r \times
%%   s}$ be matrices. The kronecker product of $F$ and $B$ is defined by

%% \begin{equation}
%% F \otimes B:=\left[
%% \begin{array}{cccc}
%% F_{0,0}B & F_{0,1}B & \cdots & F_{0,n-1}B \\
%% F_{1,0}B & F_{1,1}B & \cdots & F_{1,n-1}B \\
%% \vdots & \vdots & \ddots & \vdots\\
%% F_{m-1,0}B & F_{m-1,1}B & \cdots & F_{m-1,n-1}B 
%% \end{array}\right] \in \mathbb{R}^{mr \times ns}
%% \end{equation}
%% \end{definition}

%% One notes that $(F \otimes B)^T = F^T \otimes B^T$.
%% $F \otimes B$ can be very much larger than both $F$ and $B$ in
%% dimensions. For example, if $r = n = r = s = 1000$, so that $F$ and
%% $B$ are $10^3$-by-$10^3$ matrices, then $F \otimes B$ is a
%% $10^6$-by-$10^6$ matrix! Fortunately, given a vector $x \in
%% \mathbb{R}^{rs}$ one can compute the matvec product $(F \otimes B)x$
%% without forming $F \otimes B$. Indeed, let $y := (F \otimes B)x$. We
%% rewrite the row vector $x$ as an $s$-by-$r$ matrix $X = [X_0, X_1,
%%   ..., X_{r-1}]$, where each column vector $X_i$ is the $i$th block of
%% $s$ elements of $x$ read from left to right. Similarly, rewrite $y$ as
%% as an $n$-by-$m$ matrix $Y = [Y_0, Y_1, ..., Y_{m-1}]$. Then one
%% easily verfies that $Y$ is the matrix product of the ``small''
%% matrices $B$, $X$, and $F^T$, i.e

%% \begin{equation}
%%   \label{eq:kron_matvec}
%%   Y = BXF^T
%% \end{equation}

%% Now, in Texas Hold'em for example, the payoff matrix $A$ can be
%% written as block-diagonal matrix whose blocks are sums of Kronecker
%% products of much smaller sparse matrices as follows

%% \begin{equation}
%%   A = \begin{bmatrix}F_1 \otimes B_1\hspace{10em}\\\hspace{3em}F_2
%%     \otimes B_2\hspace{7em}\\\hspace{6em}F_3 \otimes
%%     B_3\hspace{4em}\\\hspace{10em}F_4 \otimes B_4 + S \end{bmatrix}
%% \label{eq:factor_A}
%% \end{equation}
%% See \cite{hoda2010smoothing}. The matrices $F_i$ correspond to
%% sequences of moves in round $i$ which end in a \textit{fold} action,
%% and $S$ to the sequences which end in a \textit{showdown}. $B_i$
%% encodes the betting structure of round $i$, while $W$ encodes the
%% wind/lose/draw information determined by ranking the players' hands at
%% showdown. The component matrices $F_i$, $B_i$, $S$, and $W$ are small
%% enough to be explictly represented whereas it is infeasible to
%% explicitly represent A. Furthermore, the matrices $F_i$, $B_i$,
%% $S$, and $W$ are themselves sparse, which allows one to use the
%% compressed row storage data structure that only stores nonzero entries
%% (for example \textit{scipy.sparse.csr\_matrix}, in the Python
%% programming language).

%% Such a representation of the payoff matrix $A$ trivializes matvec
%% operations involving $A$ or $A^T$ (thanks to formula
%% \eqref{eq:kron_matvec} above, applied to their diagonal blocks and
%% exploiting the sparsity of these blocks themselves). Of course, one
%% can always write the payoff matrix $A$ of a flop-type poker game in a
%% form similary to \eqref{eq:factor_A} by applying appropriate
%% permutations to the enumeration of the players's sequences.

\paragraph*{\textit(b) Computing $\|K\|$.}
Also the 2-norm $\|K\|$ of the linear operator $K$, can be efficiently
computed using the power iteration (Perron-Frobenius).

\section{Experimental results}
\label{sec:results}
To assess the practical quality of the proposed algorithm, we
tested it on simulated and real games.
\paragraph*{Experiments on sequential games with incomplete
  information: Kuhn 3-card
  poker.} This game is a simplified form of poker
developed by Harold W. Kuhn. It is a two-person zero-sum game which is
simple enough to serve as a proof-of-concept example but contains all
the complexity traits of a incomplete information sequential game. The
deck includes only three playing cards: a King, Queen, and
Jack. One card is dealt to each player, then the first player must bet
or pass, then the second player may bet or pass. If any player chooses
to bet the opposing player must bet as well ("call") in order to stay
in the round. After both players pass or bet the player with the
highest card wins the pot.%%  The
%% sequence-form representation of the game has $n_1 = n_2 = 13$
%% sequences and $l_1 = l_2 = 7$ information
%% sets per player. The sequence-form of the game is given by\\\\
%% $E_1 \in \mathbb{R}^{7 \times 13}$ with $E_1(0,0) = E_1(1,9) =
%% E_1(1,12) = E_1(2,1) = E_1(2,4) = E_1(3,5) =
%% E_1(3,8) = E_1(4,2) = E_1(4,3) = E_1(5,6) = E_1(5,7) = E_1(6,10) =
%% E_1(6,11) = {1}$, $E_1(1,0) = E_1(2,0) = E_1(3,0) = E_1(4,1) =
%% E_1(5,5) = E_1(6,9) = {-1}$; $E_2 \in \mathbb{R}^{7 \times 13}$ with
%% $E_2(0,0) = E_2(1,7) = E_2(1,8) = E_2(2,9) = E_2(2,10) = E_2(3,5) =
%% E_2(3,6) = E_2(4,11) = E_2(4,12) = E_2(5,1) = E_2(5,2) = E_2(6,3) =
%% E_2(6,4) = {1}$, $E_2(1,0) = E_2(2,0) = E_2(3,0) = E_2(4,0) =
%% E_2(5,0) = E_2(6,0) = {-1}$; and $A \in \mathbb{R}^{13 \times
%%   13}$ with $A(3,8) = A(3,12) = A(4,6) = A(4,10) = A(7,12) = A(8,10) =
%% {-0.333333}$, $A(1,7) = A(1,11) = A(2,8) = A(2,12) = A(5,11) =
%% A(6,4) = A(6,12) = A(10,4) = A(10,8) = {-0.166667}$, $A(7,4) =
%% A(8,2) = A(11,4) = A(11,8) = A(12,2) = A(12,6) = {0.333333}$,
%% $A(4,5) = A(4,9) = A(5,3) = A(8,1) = A(8,9) = A(9,3) = A(9,7) =
%% A(12,1) = A(12,5) = {0.166667}$.\\

The pair of vectors $(x^*, y^*) \in \mathbb{R}^{13 + 13}$ given by
\begin{eqnarray*}
  \begin{split}
    &x^* = [1, .759, .759, 0, .241, 1, .425, .575, 0, .275, 0,
      .275, .725]^T,\\
    &y^* = [1, 1, 0, .667, .333, .667, .333, 1, 0, 0, 1, 0, 1]^T
    \end{split}
\end{eqnarray*}
is a Nash $(10^{-4},0)$-equilibrium computed in 1500 iterations of
Algorithm  \ref{Tab:algo}. The convergence curves are shown
in Fig \ref{Tab:dgap_curve}. One easy checks that this equilibrium is
feasible. Indeed, one computes \\\\
$E_1x^* - e_1 = [4.76 \times 10^{-5}, -1.91 \times 10^{-5}, 5.67
      \times 10^{-5}, 8.23 \times 10^{-6}, 2.90 \times 10^{-5},
      -8.62 \times 10^{-7}, -1.96 \times 10^{-5}]^T$
and
$E_2y^* - e_2 = [-7.04 \times 10^{-7}, 2.27 \times 10^{-6}, -3.29
  \times 10^{-6}, -1.50 \times 10^{-6},
      2.92 \times 10^{-6}, -4.97 \times 10^{-7}, -5.85 \times
      10^{-7}]^T$.\\\\
Finally, one checks that ${x^*}^TAy^* = {-0.05555}$,
 which agrees to 5 d.p with the value of $-1 / 18$ computed
 analytically by H. W. Kuhn in his 1950 paper \cite{kuhn}. The
 evolution of the dual gap and the expected value of
 the game across iterations are shown in Figure \ref{Tab:dgap_curve}.
We have not benchmarked this against the algorithms proposed in
\cite{nesterov2005a} and Gilpin's et al. \cite{gilpinfirst} because
implementing them from scratch for such games would require us to
compute the complicated projections $\prox_{Q_k}$.  We recall that
avoiding these projections was one of the goals of the manuscript.


\paragraph*{Basic test-bed: Matrix games on simplexes.}
As in \cite{nesterov2005a,chambolle2014ergodic}, we generate a $1000
\times 1000$ random matrix whose entries are uniformly identically
distributed in the closed interval $[-1, 1]$. We compare our proposed
Algorithm \ref{Tab:algo} with Nesterov's \cite{nesterov2005a} and
Gilpin's et al. \cite{gilpinfirst}. The results of the benchmarks are
shown in Figure \ref{Tab:dgap_curve}\textit{(a)}.


\begin{figure}[!htpb]
  \subfigure[$10^3 \times 10^3$ matrix game on simplexes]{
    \includegraphics[width=.5\linewidth]{0.pdf}
  }
  %% \hspace{-2em}
  %% \subfigure[Toy poker]{
  %%   \includegraphics[width=.5\linewidth]{SimplifiedPoker_dgap.pdf}
  %% }
  \hspace{-2em}
  \subfigure[Kuhn 3-card poker]{
    \includegraphics[width=.5\linewidth]{Kuhn3112_dgap.pdf}
  }
  \vspace{-1em}
  \hspace{-.5em}
  \includegraphics[width=.5\linewidth]{1.pdf}
  %% \hspace{-1em}
  %% \includegraphics[width=.5\linewidth]{SimplifiedPoker_NE.pdf}
  %\hspace{.5em}
  \includegraphics[width=.49\linewidth]{Kuhn3112_NE.pdf}
  \caption{Convergence curves of Algorithm
    \ref{Tab:algo}. In \textit{(a)}, the duality gaps
    are computed according to formula \eqref{eq:mg_pd}. One can see
    the linear (i.e exponential) behavior of the algorithm in
    \cite{gilpinfirst}, inbetween consecutive breakpoints on the
    $\epsilon$ grid (though the rate of exponential growth seems to by
    quite close to $1$ here). The first-order smoothing algorithm from
    \cite{nesterov2005a} jitters around as the iterations go on
    because even the smoothed problem becomes heavily ill-conditioned
    near solutions. On the other hand, our proposed
    algorithm beats both of the aforementioned algorithms, and
  its proven $\mathcal{O}(1/\epsilon)$ convergence rate is clearly
  verified empirically.}
  \label{Tab:dgap_curve}
\end{figure}

\section{Concluding remarks and future work}
Making use of the sequence-form representation
\cite{koller1992complexity,von1996efficient,vonequilibrium}, we have
devised a simple and efficient primal-dual algorithm for computing
Nash equilibria in two-person zero-sum sequential games with
incomplete information (like Texas Hold'em, etc.). Our algorithm is
simple to implement, with a low constant cost per iteration, and
enjoys a rigorous convergence theory with a proven
$\mathcal{O}(1/\epsilon)$ convergence in terms of basic operations
(matvec products, clipping, etc.), to a Nash
$(\epsilon,0)$-equilibrium of the game.

Equilibrium problems are saddle-point convex-concave problems, and as
such a natural choice for algorithms for solving them would be in the
family of primal-dual algorithms. %%  We believe such primal-dual
%% schemes will receive more attention in the algorithmic game theory
%% community in future.

%% \paragraph{Software.} The authors' implementation of the proposed
%% algorithm is available upon request.

%% \paragraph{About the author:} The first author is first-year PhD
%% student in Computer Science at Universit\'e de Parix XI. My thesis
%% focuses on novel techniques for optimization on Lie groups (of
%% diffeomorphisms), and other structured manifolds, the aim being to
%% obtain better algorithms for nonlinear registration of fMRI brain
%% images and enhance the charting of human functional connectomes.

%\pagebreak
\bibliographystyle{llcns2e/splncs03}
\bibliography{bib}


%\paragraph{\textbf{Toy Poker game...}}
%% The sequence-form representation for this Toy Poker game is given by
%% (showing only nonzero entries):\\

%% $E_1 \in \mathbb{R}^{3 \times 5}$ with $E_1(0,0) = E_1(1,1) =
%% E_1(1,2) = E_1(2,3) = E_1(2,4) = {1}$,
%% $E_1(1,0) = E_1(2,0) = {-1}$; \\

%% $E_2 \in \mathbb{R}^{3 \times 5}$ with $E_2(0,0) = E_2(1,1) = E_2(1,2)
%% = E_2(2,3) = E_2(2,4) = {1}$, $E_2(1,0) = E_2(2,0) =
%% {-1}$; and \\

%% $A \in \mathbb{R}^{5 \times 5}$ with $A(2,0) =
%% A(4,0) = {-0.5}$, $A(1,3) = {1}$, $A(3,1) =
%% {-1}$, $A(1,2) = A(1,4) = A(3,2) = A(3,4) = {0.25}$.



\end{document}
